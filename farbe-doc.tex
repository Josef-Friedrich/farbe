\documentclass{ltxdoc}

\setcounter{tocdepth}{4}
\setcounter{secnumdepth}{4}

\EnableCrossrefs
\CodelineIndex
\RecordChanges

\usepackage{mdframed}
\usepackage{minted}
\usepackage{multicol}
\usepackage{luacode}
\usepackage{farbe}

\usemintedstyle{friendly}
\BeforeBeginEnvironment{minted}{\begin{mdframed}}
\AfterEndEnvironment{minted}{\end{mdframed}}
\setminted{
  breaklines=true,
  fontsize=\footnotesize,
  style=manni,
}
\def\lua#1{\mintinline{lua}|#1|}
\def\latex#1{\mintinline{latex}|#1|}

\NewDocumentCommand { \InputLatex } { O{} m } {
  \begin{mdframed}
  \inputminted[linenos=false,#1]{latex}{examples/#2}
  \end{mdframed}
}

\NewDocumentCommand { \InputLua } { O{} m } {
  \begin{mdframed}
  \inputminted[linenos=false,#1]{lua}{examples/#2}
  \end{mdframed}
}

\begin{document}

\providecommand*{\url}{\texttt}

\title{The \textsf{farbe} package}
\author{%
  Josef Friedrich\\%
  \url{josef@friedrich.rocks}\\%
  \href{https://github.com/Josef-Friedrich/farbe}
       {github.com/Josef-Friedrich/farbe}%
}
\date{0.1.0 from 2025/05/31}

\maketitle

\newpage

\tableofcontents

\newpage

% \section{Einführung}
\section{Introduction}

Color management (conversion, names) for Lua\TeX implemented in Lua.

Based on \url{https://luarocks.org/modules/Firanel/lua-color}

% CTAN stellt eine Übersichtsseite zum Thema bereit.
CTAN provides an overview page on the subject of
\href{https://www.ctan.org/topic/colour}
{Colour: packages to typesetting in colour}.

\section{base}

\directlua{farbe.print_color_table('base')}

\section{svg}

\directlua{farbe.print_color_table('svg')}

\section{x11}

\directlua{farbe.print_color_table('x11')}

%%%%%%%%%%%%%%%%%%%%%%%%%%%%%%%%%%%%%%%%%%%%%%%%%%%%%%%%%%%%%%%%%%%%%%%%
%
%%%%%%%%%%%%%%%%%%%%%%%%%%%%%%%%%%%%%%%%%%%%%%%%%%%%%%%%%%%%%%%%%%%%%%%%

\section{Lua interface / API}

\subsection{Class “Color”}

\subsubsection{Fields}

\paragraph{Color.r}

Red component.

\paragraph{Color.g}

Green component.

\paragraph{Color.b}

Blue component.

\paragraph{Color.a}

Alpha component.

\subsubsection{Methods}

\paragraph{Color:clone ()}

Clone color

\paragraph{Color:set (value)}

Set color to value.

\paragraph{Color:rgb ()}

Get rgb values.

\paragraph{Color:rgba ()}

Get rgba values.

\paragraph{Color:hsv ()}

Get hsv values.

\paragraph{Color:hsva ()}

Get hsv values.

\paragraph{Color:hsl ()}

Get hsl values.

\paragraph{Color:hsla ()}

Get hsl values.

\paragraph{Color:hwb ()}

Get hwb values.

\paragraph{Color:hwba ()}

Get hwb values.

\paragraph{Color:cmyk ()}

Get cmyk values.

\paragraph{Color:rotate (value)}

Rotate hue of color.

\paragraph{Color:invert ()}

Invert the color.

\paragraph{Color:grey ()}

Reduce saturation to 0.

\paragraph{Color:blackOrWhite (lightness)}

Set to black or white depending on lightness.

\paragraph{Color:mix (other, strength)}

Mix two colors together.

\paragraph{Color:complement ()}

Generate complementary color.

\paragraph{Color:analogous ()}

Generate analogous color scheme.

\paragraph{Color:triad ()}

Generate triadic color scheme.

\paragraph{Color:tetrad ()}

Generate tetradic color scheme.

\paragraph{Color:compound ()}

Generate compound color scheme.

\paragraph{Color:evenlySpaced (n, r)}

Generate evenly spaced color scheme.

\paragraph{Color:tostring (format)}

Get string representation of color.

\paragraph{Color:band (a, b)}

Apply rgb mask to color, providing backwards compatibility for Lua 5.1 and LuaJIT 2.1.0-beta3

\paragraph{Color:isColor (color)}

Check whether color is a Color.

\clearpage

\section{Implementation}

%%
%
%%

\subsection{farbe.lua}

\inputminted[linenos=true]{lua}{farbe.lua}

%%
%
%%

\clearpage

\subsection{farbe.tex}

\inputminted[linenos=true]{latex}{farbe.tex}

%%
%
%%

\clearpage

\subsection{farbe.sty}

\inputminted[linenos=true]{latex}{farbe.sty}

%%
%
%%

\clearpage

\changes{0.1.0}{2025/12/30}{Inital release}

\pagebreak
\PrintChanges
\pagebreak
\PrintIndex
\end{document}
